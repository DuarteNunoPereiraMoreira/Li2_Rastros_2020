\section*{Laboratórios de Informática 2}

\subsection*{Turno\+: P\+L3 Grupo\+: 4}

\subsection*{Elementos do grupo\+:}


\begin{DoxyItemize}
\item a93321 Duarte Nuno Pereira Moreira
\item a93176 Lucas Silva Carvalho
\item a93587 Marcos Paulo Ribeiro
\end{DoxyItemize}

\subsection*{1ª\+Semana}

Durante esta primeira semana de realização do projeto, o trabalho do grupo reduz-\/se em duas etapas.

Primeira etapa\+: criacao dos módulos;

Para comecar foi necessário a criação de 3 ficheiros.\+c todos com um papel distinto, ou seja, um com o objetivo de guardar informações tais como structs e outros novos tipos e/ou definições geradas por nós (dados.\+c), outro para tratar da interação jogador/jogo (interface.\+c) e por último um módulo para cuidar da parte lógica do programa (logica.\+c). Para cada um destes ficheiros.\+c foi criado um outro com extensão.\+h que contém os protótipos das funções, geradas no ficheiro.\+c correspondente, para facilitar a organização do código.

Segunda etapa\+: implementação das funções;

A dificuldade começou a surgir a partir deste ponto. Sentimos uma pequena desorientação na execução da maior parte das funções devido principalmente à falta de conhecimento sobre esta linguagem(\+C). A primeira função foi relativamente simples, uma vez que foi apenas necessário defenir os valores que constituem o estado inicial do jogo. De seguida, foi tratada a função responsável por criar parte da interface do jogo (desenhar o tabuleiro no seu estado inicial) que através de ciclos faz print dos diversos caractéres correspondentes à peça branca(\char`\"{}$\ast$\char`\"{}), às pecas pretas(\char`\"{}\#\char`\"{}) e aos espacos vazios(\char`\"{}.\char`\"{}). Por fim, foram formadas 4 pequenas e simples funçóes. Uma para alterar e outra para obter o estado de uma certa casa do tabuleiro, outra que devolve o número de jogadas executadas até ao momento e uma outra que devolve o número do jogador que deve efetuar a próxima jogada.

\subsection*{2ª\+Semana}

Na segunda semana as principais etapas foram\+:

A criacao de um prompt que dá informação aos jogadores sobre\+:

-\/O números de jogadas já efetuadas até ao momento;

-\/O número do jogador a realizar a jogada;

A implementação das jogadas\+: sendo este módulo dividido em 3 partes\+: \begin{DoxyVerb}-Verificar se uma jogada é valida, tendo em conta o estado da casa para onde o jogador pretende jogar e também se esta se  encontra na vizinhança da peça atual;
    -Modificar as funções da semana anterior para que seja colocada uma peça preta na ultima posição da peça branca;
    -Verificar se o jogo chegou ao fim, o que acabou por ser a função mais extença que fizemos, uma vez que esta cobre os casos:
    -> a casa branca encontra-se no canto inferior esquerdo: ganha o jogador 1;
        -> a casa branca encontra-se no canto superior direito: ganha o jogador 2;
    -> a jogada atual colocou a branca num sítio onde o próximo jogador não consegue jogar porque não há nenhuma casa  vizinha livre: o jogador que colocou a branca ganhou;
\end{DoxyVerb}


Por último chegou a parte de adicionar os comandos, onde surgiu o fator dificuldade, uma vez mais devido à falta de conhecimento sobre certas funções. Esta parte final levou à realização de três comandos\+: \begin{DoxyVerb}-"Q", termina com o jogo;
-"gr nome_do_ficheiro", permite ao jogador criar um ficheiro e gravar no seu interior o estado atual do tabuleiro;
-"ler nome_do_ficheiro", permite ao jogador ver o estado do tabuleiro guardado no ficheiro em questão;
\end{DoxyVerb}


\subsection*{3ª\+Semana}

Durante esta terceira fase do projeto foi-\/nos proposto\+:
\begin{DoxyItemize}
\item Fazer melhorias nos comandos ”gr” e “ler”, de maneira a que estes guardem também, informações sobre as jogadas;
\item Criação do comando movs, que tem como objetivo listar as jogadas já feitas;
\end{DoxyItemize}

Terminar o comando “gr” foi simples, uma vez que apenas precisamos de fazer com que, para além de gravar o tabuleiro, este também guardasse as jogadas efetuadas. Por outro lado, o comando “ler” mostrou-\/se um desafio pois o seu objetivo é dar procedimento a um jogo gravado a partir de um ficheiro. Isto implica que sejam recriados todos os elementos do estado do tabuleiro e da lista das jogadas, sendo utilizadas para a maioria dos casos estratégias que envolviam o número de caracteres do ficheiro.

Por fim, o comando “movs” mostrou ser também uma tarefa simples. Para o funcionamento deste comando foi necessário a criação uma função auxiliar que, a cada jogada efetuada, atualiza o arreio J\+O\+G\+A\+D\+AS de modo a que estas informações sejam mais tarde utilizadas no comando \char`\"{}movs\char`\"{}.

\subsection*{4ª\+Semana}

O trabalho relativo a esta semana do projeto foi simplesmente a implementação do comando \char`\"{}pos\char`\"{}. A execução deste novo comando, que recebe apenas um argumento (uma jogada) deve permitir ao utilizador aceder a uma jogada anterior.

O principal desafio deste comando foi, mais uma vez, a criação de uma função que alterasse o estado do jogo mas, que não se mostrou ser muito complicada. Para tal, tivemos em consideração as jogadas que já tinham sido feitas até à jogada passada como argumento e apartir dessa informação \char`\"{}recriou-\/se\char`\"{} o estado do tabuleiro, assim como, o do arreio de jogadas e as informações acerca dos jogadores.

\subsection*{5ª\+Semana}

Neste novo guião foi-\/nos proposto a criação de um módulo de listas ligadas genérico e, para alim disso, a implementação do comando \char`\"{}jog\char`\"{}. A este novo módulo foram acrescentadas várias simples funcões acerca de listas para mais tarde serem usadas no novo comando. Este deve permitir que o jogador atual peça ao computador para jogar por si. Para tal foram-\/nos fornecidas várias heurísticas da qual foi escolhida a Escolha aleatória.

A estratégia utilizada para a construção desta heurística, baseia-\/se em calcular o número de casas livres(\+N) em volta da posição atual do jogador e, posteriormente, utilizá-\/lo para formação de um número aleatório entre 0 e N. Apartir deste, basta apenas escolher a possível jogada a efetuar. 