\section*{Laboratórios de Informática 2}

\subsection*{Turno\+: P\+L3 Grupo\+: 4}

\subsection*{Elementos do grupo\+:}


\begin{DoxyItemize}
\item a93321 Duarte Nuno Pereira Moreira
\item a93176 Lucas Silva Carvalho
\item a93587 Marcos Paulo Ribeiro
\end{DoxyItemize}

\subsection*{1ª\+Semana}

Durante esta primeira semana de realização do projeto, o trabalho do grupo reduz-\/se em duas etapas.

Primeira etapa\+: criacao dos módulos;

Para comecar foi necessário a criação de 3 ficheiros.\+c todos com um papel distinto, ou seja, um com o objetivo de guardar informações tais como structs e outros novos tipos e/ou definições geradas por nós (dados.\+c), outro para tratar da interação jogador/jogo (interface.\+c) e por último um módulo para cuidar da parte lógica do programa (logica.\+c). Para cada um destes ficheiros.\+c foi criado um outro com extensão.\+h que contém os protótipos das funções, geradas no ficheiro.\+c correspondente, para facilitar a organização do código.

Segunda etapa\+: implementação das funções;

A dificuldade começou a surgir a partir deste ponto. Sentimos uma pequena desorientação na execução da maior parte das funções devido principalmente à falta de conhecimento sobre esta linguagem(\+C). A primeira função foi relativamente simples, uma vez que foi apenas necessário defenir os valores que constituem o estado inicial do jogo. De seguida, foi tratada a função responsável por criar parte da interface do jogo (desenhar o tabuleiro no seu estado inicial) que através de ciclos faz print dos diversos caractéres correspondentes à peça branca(\char`\"{}$\ast$\char`\"{}), às pecas pretas(\char`\"{}\#\char`\"{}) e aos espacos vazios(\char`\"{}.\char`\"{}). Por fim, foram formadas 4 pequenas e simples funçóes. Uma para alterar e outra para obter o estado de uma certa casa do tabuleiro, outra que devolve o número de jogadas executadas até ao momento e uma outra que devolve o número do jogador que deve efetuar a próxima jogada. 